\documentclass[a4paper, 12pt]{report}
\usepackage[a4paper,hmargin={3cm,2.5cm},vmargin={2.5cm,2.5cm}]{geometry}
\usepackage{amsfonts} % if you want blackboard bold symbols e.g. for real numbers
\usepackage{graphicx} % if you want to include jpeg or pdf pictures
\usepackage{multicol}
\usepackage{lipsum}
\usepackage{subfig}
\usepackage{float}
\usepackage{blindtext}
\usepackage{fancyhdr}
\pagestyle{fancy}
\lhead{}
\rhead{Mini Project Title}
\lfoot{Dept. of Computer Engineering}
\usepackage{tikz}
\usetikzlibrary{calc}
\usepackage{eso-pic}
\usepackage{lipsum}
\usepackage{longtable}
\begin{document}
%========================================================= %
%================begin of title page====================== %
% The frontmatter environment for everything that comes with roman numbering\
%============================================= %
\newenvironment{frontmatter}{}{}
\begin{frontmatter}
%%%%%%%%%%%%%%%%%%%%%%%%%%%%%%%%%%%%%%%%%%%%%%%%%%%%%%%%%%%%%%%%%%%
\begin{titlepage}
    %\AddToShipoutPictureBG
    \begin{center}
        \textup{\large  \textbf{Savitribai Phule Pune University}\\\textbf{Modern Education Society’s College of Engineering, Pune}}\\19, Bund Garden, V.K. Joag Path, Pune – 411001.\\[0.5cm]\textbf{ACCREDITED BY NAAC WITH “A” GRADE (CGPA – 3.13)}\\[0.5cm]\textbf{\large DEPARTMENT OF COMPUTER ENGINEERING}

        %---------------------------------Figure------------------------------

        \begin{center}
            \begin{figure}[h]  %h means here other options t , b, p, etc.
                \centering
                % \includegraphics[width=0.3\linewidth]{./logo1}
            \end{figure}
        \end{center}

        %----------------------------
        \textup{\large  A REPORT\\[0.4cm]ON}\\[0.4cm]
        \begin{LARGE}
            {\textbf {"MINI PROJECT TITLE" }}\end{LARGE}\\[1cm]
        \begin{large}\textbf {T.E. (COMPUTER)}
        \end{large}\\[0.5cm]
        \textit{SUBMITTED BY}\\[0.3cm]
        \begin{large}
            \textbf{Mr. XYZ (Exam Seat No.)}\\[0.1cm]
            \textbf{Mr. XYZ (Exam Seat No.)}\\[0.1cm]
            \textbf{Miss. XYZ (Exam Seat No.)}\\[0.1cm]
            \textbf{Miss. XYZ (Exam Seat No.)}\\[0.5cm]
        \end{large}
        \textit{UNDER THE GUIDANCE OF}\\[0.7cm]
        \begin{large}\textbf{PROF. GUIDE NAME}\\[0.3cm]\end{large}
        \textbf{(Academic Year: 2020-2021)}
        \vfill
    \end{center}
\end{titlepage}
%================begin of certificate page======================

\begin{titlepage}
    \begin{tikzpicture}[overlay,remember picture]
        \draw[line width=4pt]
        ($ (current page.north west) + (1cm,-1cm) $)
        rectangle
        ($ (current page.south east) + (-1cm,1cm) $);
        \draw[line width=1.5pt]
        ($ (current page.north west) + (1.2cm,-1.2cm) $)
        rectangle
        ($ (current page.south east) + (-1.2cm,1.2cm) $);
    \end{tikzpicture}
    \begin{center}
        \textup{\large  \textbf{Savitribai Phule Pune University}\\
            \textbf{Modern Education Society’s College of Engineering, Pune}}\\
        19, Bund Garden, V.K. Joag Path, Pune – 411001.\\[0.5cm]
        \textbf{ACCREDITED BY NAAC WITH “A” GRADE (CGPA – 3.13)}\\[0.5cm]
        \textbf{\large DEPARTMENT OF COMPUTER ENGINEERING}

        %---------------------------------Figure-----------------
        \begin{figure}[h]
            \centering
            % \includegraphics[width=0.3\linewidth]{./logo1}
        \end{figure}

        %-------------------------------------------------------
        \begin{LARGE}
            \textbf{\textit {Certificate}}\end{LARGE}\\[1.2cm]
        This is to certify that project entitled\\[0.5cm]\large\textbf{ "PROJECT TITLE HERE"}

        has been completed by \\
        Mr/Miss. Your Name ( Roll No. xxx ) \\
        Mr/Miss. Your Name ( Roll No. xxx ) \\
        Mr/Miss. Your Name ( Roll No. xxx ) \\
        Mr/Miss. Your Name ( Roll No. xxx ) \\
    \end{center}
    of TE COMP I/II/Second Shift in the Semester - I of academic year 2020-2021 in partial fulfillment of the Third Year of Bachelor degree in "Computer Engineering" as prescribed by the Savitribai Phule Pune University.
    \vspace{2cm}
    \begin{multicols}{2}
        \begin{center}
            \textbf{Prof. Guide Name\\Seminar Guide}\hspace{5cm}\\
        \end{center}
        \begin{center}
            \textbf{(Dr.(Mrs.) N. F. Shaikh)\\H.O.D}\\
        \end{center}
        \vspace{0.5cm}
    \end{multicols}
    Place: MESCOE, Pune.\\
    Date: DD/MM/2020 \\
    \vfill
\end{titlepage}
%================end of title page======================
%----------------------ACKNOWLEDGEMENT---------------------------
\pagenumbering{gobble}

% \pagebreak

% \newpage
% \begin{center}
% {\Large{\bf{\textit{ACKNOWLEDGEMENT}}\\[2cm]}}
% \end{center}
% \par \textit{It gives me great pleasure and satisfaction in presenting this mini project on “Mini Project Title”.} \\
% \par I would like to express my deep sense of gratitude towards….( Write in your own wording )\\
% \par \textit{I have furthermore to thank Computer Department HOD}  {\bf Dr.(Mrs.) N. F. Shaikh} \textit{and} {\bf Prof.guide name} \textit{to encourage me to go ahead and for continuous guidance. I also want to thank {\bf Prof. XYZ} for all her assistance and guidance for preparing report.}\\
% \par \textit{I would like to thank all those, who have directly or indirectly helped me for the completion of the work during this mini project.}
% \vspace{0.8in}
% \begin{flushright}
% {name (PRN)}\\
% {name (PRN)}\\
% {name (PRN)}\\
% {name (PRN)}\\
% \hspace*{0.1in}{T.E. Computer}\\
% \end{flushright}
% \pagenumbering{roman}
% \setcounter{page}{1}
% \newpage
% \tableofcontents
% \listoffigures
% \listoftables
% \lfoot{Modern Education Society’s College of Engineering}
% %%%%%%%%%%%%%%% Abstract %%%%%%%%%%
% %\newpage
% \begin{abstract}
% \par The first line of the first paragraph under each heading should start from left-hand margins without indentation.  Text of abstract should be typed in Times New Roman, 12pt. Keywords should be written in Times New Roman, 12pt, Italic.\\
% \par Text of chapters should be typed in Times Roman, 12pt font, normal. There should be 1 blank spaces gap between the subsequent sentences. For the Seminar report, the maximum length should not exceed 35 pages, exclusive of all figures, tables, captions, notes and references. 
% \\
% \vspace{0.3cm}

% \textbf{Keywords-}\it{\textbf{Orientation Field Estimation, Enhanced Feedback, Minutiae, Fuzzy Feature Match(FFM).}}\\
% \end{abstract}
% %================================================ %
% % The frontmatter environment for everything that comes with roman numbering %
% \end{frontmatter}
% %%%%%%%%%%%%%%%%%%%%%%%%%%%%%%%%%%%%%%%

% \newpage
% \pagenumbering{arabic}
% %%%%%%%%% MAIN TEXT STARTS HERE %%%%%%%%%%
% \chapter{INTRODUCTION}
% \section{Introduction}
% \par The first line of the first paragraph under each heading should start from left-hand margins without indentation.  Text of abstract should be typed in Times New Roman, 12pt. Keywords should be written in Times New Roman, 12pt, Italic.\\
% \section{Motivation}
% \par The first line of the first paragraph under each heading should start from left-hand margins without indentation.  Text of abstract should be typed in Times New Roman, 12pt. Keywords should be written in Times New Roman, 12pt, Italic.\\
% \chapter{PROBLEM  STATEMENT}


% \section{Problem Statement}
% \par The first line of the first paragraph under each heading should start from left-hand margins without indentation.  Text of abstract should be typed in Times New Roman, 12pt. Keywords should be written in Times New Roman, 12pt, Italic.\\
% \section{Explanation}
% \par The first line of the first paragraph under each heading should start from left-hand margins without indentation.  Text of abstract should be typed in Times New Roman, 12pt. Keywords should be written in Times New Roman, 12pt, Italic.\\
% \begin{table}[h]

% \begin{center}
%  \begin{tabular}{||c|c|c|c||} 

%  \hline
%  Col1 & Col2 & Col2 & Col3 \\ [0.5ex] 
%  \hline\hline
%  1 & 6 & 87837 & 787 \\ 
%  \hline
%  2 & 7 & 78 & 5415 \\
%  \hline
%  3 & 545 & 778 & 7507 \\
%  \hline
%  4 & 545 & 18744 & 7560 \\
%  \hline
%  5 & 88 & 788 & 6344 \\ [1ex] 
%  \hline
% \end{tabular}\caption{table name}
% \end{center}


% \end{table}
% \chapter{SOFTWARE REQUIREMENT SPECIFICATION}
% \section{Software and Hardware Requirement}
% \par The first line of the first paragraph under each heading should start from left-hand margins without indentation.  Text of abstract should be typed in Times New Roman, 12pt. Keywords should be written in Times New Roman, 12pt, Italic.\\

% \subsection{Description}
% \par The first line of the first paragraph under each heading should start from left-hand margins without indentation.  Text of abstract should be typed in Times New Roman, 12pt. Keywords should be written in Times New Roman, 12pt, Italic.\\
% \section{ER Diagram}
% \begin{figure}[h]
% \centering
% \includegraphics[width=0.3\linewidth]{./logo1}
% \caption{fig name}
% \end{figure}
% \par The first line of the first paragraph under each heading should start from left-hand margins without indentation.  Text of abstract should be typed in Times New Roman, 12pt. Keywords should be written in Times New Roman, 12pt, Italic.\\

% \section{Database Connectivity}
% \par The first line of the first paragraph under each heading should start from left-hand margins without indentation.  Text of abstract should be typed in Times New Roman, 12pt. Keywords should be written in Times New Roman, 12pt, Italic.\\
% \chapter{IMPLEMENTATION OF PROJECT}
% \section{Modules description}
% \par The first line of the first paragraph under each heading should start from left-hand margins without indentation.  Text of abstract should be typed in Times New Roman, 12pt. Keywords should be written in Times New Roman, 12pt, Italic.\\
% \section{Screen shots of Project}
% \begin{figure}[h]
% \centering
% \includegraphics[width=0.3\linewidth]{./logo1} %Change This Image
% \caption{fig name}
% \end{figure}
% \chapter{CONCLUSION}
% \par The first line of the first paragraph under each heading should start from left-hand margins without indentation.  Text of abstract should be typed in Times New Roman, 12pt. Keywords should be written in Times New Roman, 12pt, Italic.\\
% \end{document} 