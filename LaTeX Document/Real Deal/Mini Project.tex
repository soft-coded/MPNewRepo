\documentclass[a4paper]{article}
\usepackage[a4paper,hmargin={3cm,2.5cm},vmargin={2.5cm,2.5cm}]{geometry}

\usepackage{tikz}
\usetikzlibrary{calc}

\usepackage{fancyhdr}
\pagestyle{fancy}
\lhead{}
\rhead{\textcolor{red}{Mini Project Title}}
\lfoot{Department of Computer Engineering}

\usepackage{lipsum}

\usepackage{multicol}

\usepackage{ragged2e}

\usepackage{xcolor}

\usepackage{enumitem}

\begin{document}
\begin{titlepage}
    \begin{tikzpicture}[overlay,remember picture]
        \draw[line width=4pt]
        ($ (current page.north west) + (1cm,-1cm) $)
        rectangle
        ($ (current page.south east) + (-1cm,1cm) $);
        \draw[line width=1.5pt]
        ($ (current page.north west) + (1.2cm,-1.2cm) $)
        rectangle
        ($ (current page.south east) + (-1.2cm,1.2cm) $);
    \end{tikzpicture}

    \begin{center}

        \textup{\large  \textbf{INDIAN INSTITUTE OF ENGINEERING}\\\textbf{SCIENCE AND TECHNOLOGY, SHIBPUR}}\\ Howrah, West Bengal, India - 711103\\[0.5cm]\textbf{\large DEPARTMENT OF COMPUTER SCIENCE}\\\textbf{\large AND TECHNOLOGY}

        %---------------------------------Figure------------------------------
        \begin{center}
            \begin{figure}[h]   %h means here other options t , b, p, etc.
                \centering
                \includegraphics[width=0.3\linewidth]{Pictures/IIESTS Logo.png}
            \end{figure}
        \end{center}

        %----------------------------

        \textup{\large A MINI-PROJECT REPORT SUBMITTED IN PARTIAL FULFILLMENT OF THE REQUIREMENTS\\[0.4cm]ON}\\[0.4cm]

        \begin{LARGE}
            {\textbf {\textcolor{red}{"MINI PROJECT TITLE"}}}
        \end{LARGE}\\[1cm]

        \textit{SUBMITTED BY}\\[0.3cm]
        \begin{large}
            \textbf{Abhinaba Chowdhury (510519007)}\\[0.1cm]
            \textbf{Abhiroop Mukherjee (510510109)}\\[0.1cm]
            \textbf{Debarghya Dey (510519087)}\\[0.1cm]
            \textbf{Jyotiprakash Roy (510519016)}\\[0.1cm]
            \textbf{Shrutanten (510519048)}\\[1cm]
        \end{large}

        \textit{UNDER THE GUIDANCE OF}\\[0.5cm]
        \begin{large}
            \textbf{DR. SAMIT BISWAS}\\[0.3cm]
        \end{large}

        \textbf{(Academic Year: 2020-2021)}
    \end{center}
\end{titlepage}
%================begin of certificate page======================

\begin{titlepage}
    \begin{tikzpicture}[overlay,remember picture]
        \draw[line width=4pt]
        ($ (current page.north west) + (1cm,-1cm) $)
        rectangle
        ($ (current page.south east) + (-1cm,1cm) $);
        \draw[line width=1.5pt]
        ($ (current page.north west) + (1.2cm,-1.2cm) $)
        rectangle
        ($ (current page.south east) + (-1.2cm,1.2cm) $);
    \end{tikzpicture}

    \begin{center}
        \textup{\large  \textbf{INDIAN INSTITUTE OF ENGINEERING}\\\textbf{SCIENCE AND TECHNOLOGY, SHIBPUR}}\\ Howrah, West Bengal, India - 711103\\[0.5cm]\textbf{\large DEPARTMENT OF COMPUTER SCIENCE}\\\textbf{\large AND TECHNOLOGY}

        %---------------------------------Figure------------------------------
        \begin{center}
            \begin{figure}[h]   %h means here other options t , b, p, etc.
                \centering
                \includegraphics[width=0.3\linewidth]{Pictures/IIESTS Logo.png}
            \end{figure}
        \end{center}

        %----------------------------


        \begin{LARGE}
            \textbf{\textit {Certificate}}
        \end{LARGE}\\[1.2cm]
    \end{center}

    It is certified hereby that this report, titled *whatever the title is*, and all the attached
    documents herewith are authentic records of Abhinaba Chowdhury (510519007),\\Abhiroop Mukherjee (510510109), Debarghya Dey (510519087),
    Jyotiprakash Roy \\(510519016), and Shrutanten (510519048) from the Prestigious Department of \\Computer Science And Technology
    of the Distinguished and Respected IIEST Shibpur under my guidance.

    The works of these students are satisfies all the requirements for which it is submitted.
    To the extent of my knowledge, it has not been submitted to any different institutions for
    the awards of degree/diploma.

    \vspace{5cm}
    \begin{multicols}{2}
        \begin{center}
            \textbf{Dr. Samit Biswas\\Asst. Professor}\hspace{5cm}
        \end{center}
        \begin{center}
            \textbf{Dr. Sekhar Mandal\\Head Of Department}
        \end{center}
        \vspace{0.5cm}
    \end{multicols}
    \vfill
\end{titlepage}
%================end of title page======================

%----------------------ACKNOWLEDGEMENT---------------------------
\pagebreak
\newpage

\begin{titlepage}
    \begin{center}
        {\Large{\bf{\textit{ACKNOWLEDGEMENT}}\\[2cm]}}
    \end{center}


    \paragraph{\normalfont\textit{\indent We, as the students of IIEST, consider ourselves honoured to be working with Dr. Samit Biswas.
            The success of this project would not have been possible without his useful insights,
            appropriate guidance and necessary criticism.}}
    \paragraph{\normalfont\textit{\indent We would pass our token of token of gratitude to the Department of Computer Science And Technoogy as well for providing
            us with the opportunity to be able to tackle real world problems while improving
            our problem solving ability and thinking capacity by organising this project. We all have
            learnt quite a handful of new skills and are eager to use them henceforth as well.}}

    \begin{flushleft}
        \textit{Abhinaba Chowdhury (510519007)\\
            Abhiroop Mukherjee (510510109)\\
            Debarghya Dey (510519087)\\
            Jyotiprakash Roy (510519016)\\
            Shrutanten (510519048)}

    \end{flushleft}
\end{titlepage}

\pagenumbering{roman}
\setcounter{page}{1}
\newpage
\setcounter{tocdepth}{2} % + subsections
\tableofcontents
\lfoot{IIEST, Shibpur}
\newpage

\pagenumbering{arabic}
\setcounter{page}{1}

%%%%%%%%% MAIN TEXT STARTS HERE %%%%%%%%%%
\section{INTRODUCTION}
\subsection{Motivation}
Coronaviruses are a group of related RNA viruses that cause diseases in mammals
and birds. In humans and birds, they cause respiratory tract infections that can
range from mild to lethal. Mild illnesses in humans include some cases of the
common cold (which is also caused by other viruses, predominantly rhinoviruses),
while more lethal varieties can cause SARS, MERS, and COVID-19.

With the increase in the spread of the dangerous and highly contagious \textbf{Novel Coronavirus}
and the underlying disease caused by it, \textbf{COVID-19},
it is a requirement now more than ever to follow the social distancing
norms set in place by the scientists and researchers.

But as we all know, India is a country with a not-so-small population,
so it is pretty understandable and obvious that the law enforcement will
not be able to actually enforce it on every single person. Therefore,
new means of automata in place of actual individuals is a no brainier.

That is where we come in.

\subsection{The Idea Behind The Project}
The idea behind the working of this software was simple. The software just needed
to be able to look at a live feed (or recorded footage) of a camera and know
which of the people present in the footage are actually following the social
distancing norms and which of them are not, and mark either one appropriately.
That is where out journey to build a social distance checker started.

\textcolor{red}{// will add more later probably lul}

\newpage

\section{PREREQUISITES}

\subsection{Outdoor Requirements}
It is important to mention here that this is not a portable software that can
be fed any footage and just be expected to work. There need to be some
calibration measures taken to actually get this software working:

\begin{itemize}
    \item Actually knowing the local social distancing norms
          \begin{itemize}
              \item The minimum distance set for social distancing by the local government
          \end{itemize}

    \item Finding a good position for the camera
          \begin{itemize}
              \item The footage needs to be taken from a high enough place
          \end{itemize}

    \item Knowing the required distance in pixels
          \begin{itemize}
              \item This will depend on the position and angle of the camera's view
          \end{itemize}
\end{itemize}

\subsection{Hardware and Software Requirements}
The tools used to build this software are platform independent. However,
there are a few requirements needed to be fulfilled to get the program
working. These are:

\begin{itemize}
    \item Software Requirements
          \begin{itemize}
              \item Python - 3.5 or above
              \item OpenCV-Python - version 2 or above
              \item YOLOv3 Configuration and Network Weights
              \item Numpy
          \end{itemize}

    \item Hardware Requirements
          \begin{itemize}
              \item A GPU is optional yet recommended to get the best performance
              \item If a GPU is not being used, the CPU need to be good enough
          \end{itemize}
\end{itemize}

\section{THE PROJECT}
\subsection{Software Used}
The softwares used to build this \textit{checker} are:

\subsubsection{An Integrated Development Environment (IDE)}
An integrated development environment (IDE) is a software application that
provides comprehensive facilities to computer programmers for software
development. An IDE normally consists of at least a source code editor, build
automation tools and a debugger. Some IDEs contain the necessary compiler,
interpreter, or both; others, do not.

\subsubsection{Python}
Python is an interpreted, high-level and general-purpose programming language.
Python's design philosophy emphasizes code readability with its notable use of
significant whitespace. Its language constructs and object-oriented approach aim
to help programmers write clear, logical code for small and large-scale projects.

\paragraph{Why did we choose Python?}
\begin{enumerate}
    \item Python has an upper hand when it comes to software based on
    image recognition and object detection. Since it is the main
    objective of the project, choosing python was a given.Python has an upper hand when it comes to software based on
    image recognition and object detection. Since it is the main
    objective of the project, choosing python was a given.
    \item Python is unbeaten when it comes to Machine Learning. Python has
    support for myriad machine learning libraries, such as OpenCV, the
    one being used here.
    \item Python is comparatively easier to understand and learn. The syntax
    is clear and simple to read and write.
    \item And just our overall experience of using python for years.
\end{enumerate}

\end{document}